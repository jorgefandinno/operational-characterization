%----------------------------------------------------------------
\begin{frame}{Unfounded sets}

  Let $P$ be a normal logic program,\\ and
  let $\langle T,F \rangle$ be a partial interpretation

  \begin{itemize}
  \item<2-> A set $U \subseteq \atom{P}$ is an \alert{unfounded set} of $P$
    wrt $\langle T,F \rangle$\pause[4],

%    if we have for each rule $r\in P$ such that $\head{r}\in U$\pause[5] either
    if for each rule $r\in P$ such that $\head{r}\in U$\pause[5], we have that

    \begin{enumerate}
      % \item $\head{r} \not\in U$,
    \item<6-> $\pbody{r} \cap F \neq \emptyset$ or $\nbody{r} \cap T \neq \emptyset$\pause[8] or
    \item<8-> $\pbody{r} \cap U \neq \emptyset$
    \end{enumerate}

  \item<3> Intuitively, $\langle T,F \rangle$ is what we already know about $P$
  \item<7> Rules satisfying Condition 1 are not usable for further derivations
  \item<9-> Condition 2 is the unfounded set condition treating cyclic derivations:
    \alert{All rules still being usable to derive an atom in $U$ require an(other) atom in $U$ to be true}
  \end{itemize}
\end{frame}
%----------------------------------------------------------------
\begin{frame}{Example}
%
\[
P
=
\left\{
  \begin{array}{lcl}
    a &\leftarrow & b \\
    b &\leftarrow & a
  \end{array}
\right\}
\]
\begin{itemize}
\item<2-> $\emptyset$ is an unfounded set (by definition)
\smallskip
\item<3-> $\{a\}$ is not an unfounded set of $P$ wrt $\langle\emptyset,\emptyset\rangle$
\item<4-> $\{a\}$ is     an unfounded set of $P$ wrt $\langle\emptyset,\{b\}\rangle$
\item<5-> $\{a\}$ is not an unfounded set of $P$ wrt $\langle\{b\},\emptyset\rangle$
\smallskip
\item<6> Analogously for $\{b\}$
\smallskip
\item<7-> $\{a,b\}$ is     an unfounded set of $P$ wrt $\langle\emptyset,\emptyset\rangle$
\item<8-> $\{a,b\}$ is     an unfounded set of $P$ wrt any partial interpretation
\end{itemize}

\end{frame}
%----------------------------------------------------------------
\begin{frame}{Unfounded sets an loops}

\end{frame}
% ----------------------------------------------------------------
\begin{frame}{Greatest unfounded sets}\label{unf:greatest}

Let $P$ be a normal logic program,\\ and let
$\langle T,F \rangle$ be a partial interpretation

\begin{itemize}
\item<2-> \structure{Observation} \ The union of two unfounded sets is an unfounded set
\item<3-> The \alert{greatest unfounded set} of $P$ wrt $\langle T,F \rangle$ is the
  union of all unfounded sets of $P$ wrt $\langle T,F \rangle$
\item<4-> [] It is denoted by $\mathbf{U}_P\langle T,F \rangle$
\item<5-> Alternatively, we may define
  \[
  \mathbf{U}_P\langle T,F \rangle =
  \atom{P} \setminus
  \Cn{\reduct{\{r \in P \mid \pbody{r} \cap F = \emptyset\}}{T}}\/
  \]
\item<6-> \structure{Note} \
  \(
  \Cn{\reduct{\{r \in P \mid \pbody{r} \cap F = \emptyset\}}{T}}
  \)
  contains all non-circularly derivable atoms from $P$ wrt $\langle T,F \rangle$
\end{itemize}
\end{frame}
% ----------------------------------------------------------------------
%
%%% Local Variables:
%%% mode: latex
%%% TeX-master: "../../main"
%%% End:
