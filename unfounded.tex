%----------------------------------------------------------------
\begin{frame}{Unfounded sets}

  Let $P$ be a normal logic program,\\ and
  let $\langle T,F \rangle$ be a partial interpretation
  \smallskip
  \begin{itemize}
  \item<2-> A set $U \subseteq \atom{P}$ is an \alert{unfounded set} of $P$
    wrt $\langle T,F \rangle$
  \item<4->[]
  % if for each rule $r\in P$ such that $\head{r}\in U$ either
    if for each rule $r\in P$ such that $\head{r}\in U$

    \begin{enumerate}\normalsize
  % \item $\head{r} \not\in U$,
    \item<5-> $\pbody{r} \cap F \neq \emptyset$ or $\nbody{r} \cap T \neq \emptyset$\uncover<7->{ or}
    \item<7-> $\pbody{r} \cap U \neq \emptyset$
    \end{enumerate}
    \medskip
  \item<3-> \structure{Note} \
    \begin{itemize}\normalsize
    \item<3-> $\langle T,F \rangle$ is what we already know about $P$
    \item<6-> Condition 1 identifies inapplicable rules
    \item<8-> Condition 2 identifies cyclic derivations
      \smallskip
    \item<9->[] \structure{Unfounded set condition} \
    \alert{All rules still being usable\\ to derive an atom in $U$ require an atom in $U$ to be true}
  \end{itemize}
\end{itemize}
\end{frame}
%----------------------------------------------------------------
\begin{frame}{Example}
%
\[
P
=
\left\{
  \begin{array}{lcl}
    a &\leftarrow & b \\
    b &\leftarrow & a
  \end{array}
\right\}
\]
\medskip
\begin{itemize}
\item<2-> $\emptyset$ is an unfounded set (by definition)
\smallskip
\item<3-> $\{a\}$ is not an unfounded set of $P$ wrt $\langle\emptyset,\emptyset\rangle$
\item<4-> $\{a\}$ is     an unfounded set of $P$ wrt $\langle\emptyset,\{b\}\rangle$
\item<5-> $\{a\}$ is not an unfounded set of $P$ wrt $\langle\{b\},\emptyset\rangle$
\smallskip
\item<6>[]\itarrow\ analogously for $\{b\}$
\smallskip
\item<7-> $\{a,b\}$ is     an unfounded set of $P$ wrt $\langle\emptyset,\emptyset\rangle$
\item<8-> $\{a,b\}$ is     an unfounded set of $P$ wrt any partial interpretation
\end{itemize}

\end{frame}
% ----------------------------------------------------------------------
\begin{frame}{Logic programs}
  \bigskip
  \begin{center}
    \begin{minipage}[t]{0.92\linewidth}
      \begin{block}{Sacc{\'a}-Zaniolo Theorem}
        Let $P$ be a normal logic program and $X\subseteq\atom{P}$
        \par\medskip
        Then, $X$ is a stable model of~$P$ iff
        \smallskip
        \begin{enumerate}
        \item $X\models P$ and
        \item no nonempty subset of $X$ is unfounded for $P$ wrt $\langle X,\atom{P}\setminus X \rangle$
        \end{enumerate}
      \end{block}
    \end{minipage}
  \end{center}
  \nocite{saczan90a}
\end{frame}
%----------------------------------------------------------------
\begin{frame}{Unfounded sets an loops}

\end{frame}
% ----------------------------------------------------------------
\begin{frame}{Greatest unfounded sets}\label{unf:greatest}

Let $P$ be a normal logic program,\\ and let
$\langle T,F \rangle$ be a partial interpretation

\begin{itemize}
\item<2-> \structure{Observation} \ The union of two unfounded sets is an unfounded set
\item<3-> The \alert{greatest unfounded set} of $P$ wrt $\langle T,F \rangle$ is the
  union of all unfounded sets of $P$ wrt $\langle T,F \rangle$
\item<4-> [] It is denoted by $\mathbf{U}_P\langle T,F \rangle$
\item<5-> Alternatively, we may define
  \[
  \mathbf{U}_P\langle T,F \rangle =
  \atom{P} \setminus
  \Cn{\reduct{\{r \in P \mid \pbody{r} \cap F = \emptyset\}}{T}}\/
  \]
\item<6-> \structure{Note} \
  \(
  \Cn{\reduct{\{r \in P \mid \pbody{r} \cap F = \emptyset\}}{T}}
  \)
  contains all non-circularly derivable atoms from $P$ wrt $\langle T,F \rangle$
\end{itemize}
\end{frame}
% ----------------------------------------------------------------------
%
%%% Local Variables:
%%% mode: latex
%%% TeX-master: "../../main"
%%% End:
