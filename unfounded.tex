%------------------------------------------------------------
\section{Unfounded Sets}
%----------------------------------------------------------------
\begin{frame}{Unfounded sets}

  Let $P$ be a normal logic program,\\ and
  let $\langle T,F \rangle$ be a partial interpretation

  \begin{itemize}
  \item<2-> A set $U \subseteq \atom{P}$ is an \alert{unfounded set} of $P$
    wrt $\langle T,F \rangle$\pause[4],

%    if we have for each rule $r\in P$ such that $\head{r}\in U$\pause[5] either
    if for each rule $r\in P$ such that $\head{r}\in U$\pause[5], we have that

    \begin{enumerate}
      % \item $\head{r} \not\in U$,
    \item<6-> $\pbody{r} \cap F \neq \emptyset$ or $\nbody{r} \cap T \neq \emptyset$\pause[8] or
    \item<8-> $\pbody{r} \cap U \neq \emptyset$
    \end{enumerate}

  \item<3> Intuitively, $\langle T,F \rangle$ is what we already know about $P$
  \item<7> Rules satisfying Condition 1 are not usable for further derivations
  \item<9-> Condition 2 is the unfounded set condition treating cyclic derivations:
    \alert{All rules still being usable to derive an atom in $U$ require an(other) atom in $U$ to be true}
  \end{itemize}
\end{frame}
%----------------------------------------------------------------
\begin{frame}{Example}
%
\[
P
=
\left\{
  \begin{array}{lcl}
    a &\leftarrow & b \\
    b &\leftarrow & a
  \end{array}
\right\}
\]
\begin{itemize}
\item<2-> $\emptyset$ is an unfounded set (by definition)
\smallskip
\item<3-> $\{a\}$ is not an unfounded set of $P$ wrt $\langle\emptyset,\emptyset\rangle$
\item<4-> $\{a\}$ is     an unfounded set of $P$ wrt $\langle\emptyset,\{b\}\rangle$
\item<5-> $\{a\}$ is not an unfounded set of $P$ wrt $\langle\{b\},\emptyset\rangle$
\smallskip
\item<6> Analogously for $\{b\}$
\smallskip
\item<7-> $\{a,b\}$ is     an unfounded set of $P$ wrt $\langle\emptyset,\emptyset\rangle$
\item<8-> $\{a,b\}$ is     an unfounded set of $P$ wrt any partial interpretation
\end{itemize}

\end{frame}
%----------------------------------------------------------------
\begin{frame}{Greatest unfounded sets}\label{unf:greatest}

Let $P$ be a normal logic program,\\ and let
$\langle T,F \rangle$ be a partial interpretation

\begin{itemize}
\item<2-> \structure{Observation} The union of two unfounded sets is an unfounded set
\item<3-> The \alert{greatest unfounded set} of $P$ wrt $\langle T,F \rangle$ is the
  union of all unfounded sets of $P$ wrt $\langle T,F \rangle$
\item<4-> [] It is denoted by $\mathbf{U}_P\langle T,F \rangle$
\item<5-> Alternatively, we may define
  \[
  \mathbf{U}_P\langle T,F \rangle =
  \atom{P} \setminus
  \Cn{\reduct{\{r \in P \mid \pbody{r} \cap F = \emptyset\}}{T}}\/
  \]
\item<6-> \structure{Note}
  \(
  \Cn{\reduct{\{r \in P \mid \pbody{r} \cap F = \emptyset\}}{T}}
  \)
  contains all non-circularly derivable atoms from $P$ wrt $\langle T,F \rangle$
\end{itemize}
\end{frame}
%------------------------------------------------------------
\section{Well-Founded Operator}
%----------------------------------------------------------------
\begin{frame}{Well-founded operator}

  Let $P$ be a normal logic program,\\
  and let $\langle T,F \rangle$ be a partial interpretation

  \begin{itemize}
  \item<1-> \structure{Observation}
    Condition 1 (in the definition of an unfounded set)
    corresponds to  ${\mathbf{F}}_P\langle T,F \rangle$
    of Fitting's ${\mathbf{\Phi}}_P\langle T,F \rangle$
  \item<2-> \structure{Idea}
    Extend (negative part of) Fitting's operator $\mathbf{\Phi}_P$

  \item<3-> [] That is,
    \begin{itemize}
    \item keep definition of ${\mathbf{T}}_P\langle T,F \rangle$ from ${\mathbf{\Phi}}_P\langle T,F \rangle$ and
    \item replace ${\mathbf{F}}_P\langle T,F \rangle$ from ${\mathbf{\Phi}}_P\langle T,F \rangle$ by
      ${\mathbf{U}}_P\langle T,F \rangle$
    \end{itemize}
  \item<4-> In words, an atom must be $\mathit{false}$\\ if it belongs to the greatest unfounded set
    \bigskip
  \item<5-> \structure{Definition}
    \(
    {\mathbf{\Omega}}_P\langle T,F \rangle
    =
    \langle {\mathbf{T}}_P\langle T,F \rangle, {\mathbf{U}}_P\langle T,F \rangle\rangle
    \)
    \pause
  \item<5-> \structure{Property}
    \(
    {\mathbf{\Phi}}_P\langle T,F \rangle \sqsubseteq
    {\mathbf{\Omega}}_P\langle T,F \rangle
    \)
  \end{itemize}
\end{frame}
%----------------------------------------------------------------
\begin{frame}{Example}
  \begin{itemize}
  \item<1-> []
    \[
    P
    =
    \left\{
      \begin{array}{lll}
        a \leftarrow                  \quad &
        c \leftarrow a, \naf{d}       \quad &
        e \leftarrow b, \naf{f}
        \\
        b \leftarrow \naf{a}          \quad &
        d \leftarrow \naf{c}, \naf{e} \quad &
        e \leftarrow e
      \end{array}
    \right\}
    \]
    \medskip
  \item<2-> Let's iterate ${\mathbf{\Omega}}_{P_1}$ on $\langle \{c\}, \emptyset\rangle$:
    \[
    \begin{array}{rcl}
      {\mathbf{\Omega}}_{P}\langle\{c\}  ,\emptyset\rangle &=&\langle\{a\}  ,\{d,f\}     \rangle
      \\
      {\mathbf{\Omega}}_{P}\langle\{a\}  ,\{d,f\}    \rangle &=&\langle\{a,c\},\{b,e,f\}   \rangle
      \\
      {\mathbf{\Omega}}_{P}\langle\{a,c\},\{b,e,f\}  \rangle &=&\langle\{a\}  ,\{b,d,e,f\} \rangle
      \\
      {\mathbf{\Omega}}_{P}\langle\{a\}  ,\{b,d,e,f\}\rangle &=&\langle\{a,c\},\{b,e,f\}   \rangle
      \\
      & \vdots &
    \end{array}
    \]
  \end{itemize}
\end{frame}
%----------------------------------------------------------------
\begin{frame}{Well-founded semantics}
  \bigskip
  \begin{itemize}
  \item<1-> Define the iterative variant of ${\mathbf{\Omega}}_P$ analogously to $\mathbf{\Phi}_P$:
    \[
    {\mathbf{\Omega}}_P^0\langle T, F \rangle = \langle T, F \rangle
    \qquad\qquad\qquad
    {\mathbf{\Omega}}_P^{i+1}\langle T, F \rangle =
    {\mathbf{\Omega}}_P{\mathbf{\Omega}}_P^i\langle T, F \rangle
    \]
  \item<2-> Define the \alert{well-founded semantics} of a normal logic program $P$\\
    as the partial interpretation:
    \[
    \textstyle{\bigsqcup_{i \geq 0}} {\mathbf{\Omega}}_P^i \langle \emptyset, \emptyset \rangle
    \]
  \end{itemize}
\end{frame}
%----------------------------------------------------------------
\begin{frame}{Example}
  \begin{itemize}
  \item<1-> []
    \[
    P
    =
    \left\{
      \begin{array}{lll}
        a \leftarrow                  \quad &
        c \leftarrow a, \naf{d}       \quad &
        e \leftarrow b, \naf{f}
        \\
        b \leftarrow \naf{a}          \quad &
        d \leftarrow \naf{c}, \naf{e} \quad &
        e \leftarrow e
      \end{array}
    \right\}
    \]

  \item<2-> []
    \[
    \begin{array}{rclcl}
      \mathbf{\Omega}^0\langle\emptyset,\emptyset\rangle&=&                                                & &\langle\emptyset,\emptyset\rangle
      \\
      \mathbf{\Omega}^1\langle\emptyset,\emptyset\rangle&=&\mathbf{\Omega}\langle\emptyset,\emptyset\rangle&=&\langle\{a\}    ,\{f\}    \rangle
      \\
      \mathbf{\Omega}^2\langle\emptyset,\emptyset\rangle&=&\mathbf{\Omega}\langle\{a\}    ,\{f\}    \rangle&=&\langle\{a\}    ,\{b,e,f\}\rangle
      \\
      \mathbf{\Omega}^3\langle\emptyset,\emptyset\rangle&=&\mathbf{\Omega}\langle\{a\}    ,\{b,e,f\}\rangle&=&\langle\{a\}    ,\{b,e,f\}\rangle
      \\[10pt]
      \bigsqcup_{i \geq 0}\mathbf{\Omega}^i\langle\emptyset,\emptyset\rangle&=&\langle\{a\},\{b,e,f\}\rangle
    \end{array}
    \]
  \end{itemize}
\end{frame}
%----------------------------------------------------------------
\begin{frame}{Properties}
  \bigskip
  Let $P$ be a normal logic program
  \medskip
  \begin{itemize}
  \item ${\mathbf{\Omega}}_P\langle \emptyset, \emptyset \rangle$ is monotonic

    That is,
    \(
    {\mathbf{\Omega}}_P^i\langle \emptyset, \emptyset \rangle
    \sqsubseteq
    {\mathbf{\Omega}}_P^{i+1}\langle \emptyset, \emptyset \rangle
    \)
    \smallskip
  \item The well-founded semantics of $P$ is
    \begin{itemize}
    \item not conflicting,
    \item and generally not total
    \end{itemize}
  \item We have
    \(
    \bigsqcup_{i \geq 0}
    {\mathbf{\Phi}}_P^i \langle \emptyset, \emptyset \rangle
    \sqsubseteq
    \bigsqcup_{i \geq 0} {\mathbf{\Omega}}_P^i \langle
    \emptyset, \emptyset \rangle
    \)
  \end{itemize}
\end{frame}
% ----------------------------------------------------------------
\begin{frame}{Well-founded fixpoints}
  \bigskip
  Let $P$ be a normal logic program,\\ and
  let $\langle T,F \rangle$ be a partial interpretation
  \medskip
  \begin{itemize}
  \item<1-> Define
    $\langle T,F \rangle$ as a \alert{well-founded fixpoint} of $P$
    if
    ${\mathbf{\Omega}}_P\langle T,F \rangle = \langle T,F \rangle$
  \medskip
  \item<2-> []
    \begin{itemize}
    \item The well-founded semantics is the $\sqsubseteq$-least well-founded
      fixpoint of~$P$
    \item Any other well-founded fixpoint extends the well-founded semantics
    \item Total well-founded fixpoints correspond to stable models
    \end{itemize}
  \end{itemize}
\end{frame}
%----------------------------------------------------------------
\begin{frame}{Well-founded fixpoints: Example}
  \bigskip
  \begin{itemize}
  \item<1-> []
    \[
    P
    =
    \left\{
      \begin{array}{lll}
        a \leftarrow                  \quad &
        c \leftarrow a, \naf{d}       \quad &
        e \leftarrow b, \naf{f}
        \\
        b \leftarrow \naf{a}          \quad &
        d \leftarrow \naf{c}, \naf{e} \quad &
        e \leftarrow e
      \end{array}
    \right\}
    \]
    \bigskip
  \item<2-> $P$ has two total well-founded fixpoints:
    \begin{enumerate}
    \item<3-> $\langle \{a,c\}, \{b,d,e,f\} \rangle$
    \item<3-> $\langle \{a,d\}, \{b,c,e,f\} \rangle$
    \end{enumerate}
    \medskip
  \item <4-> Both of them represent stable models
  \end{itemize}
\end{frame}
%----------------------------------------------------------------
\begin{frame}{Properties}

  Let $P$ be a normal logic program,\\ and
  let $\langle T,F \rangle$ be a partial interpretation

  \begin{itemize}
  \item<1-> Let ${\mathbf{\Omega}}_P\langle T,F \rangle=\langle T',F'\rangle$
  \item<2-> If $X$ is a stable model of $P$ such that $T\subseteq X$ and $X\cap F=\emptyset$,
    \par then $T'\subseteq X$ and $X\cap F'=\emptyset$
  \item<3-> [] That is, ${\mathbf{\Omega}}_P$ is \alert{stable model preserving}
  \item<3-> [] Hence, ${\mathbf{\Omega}}_P$ can be used for approximating stable models
    and so for propagation in ASP-solvers
    \medskip
  \item <4-> In contrast to ${\mathbf{\Phi}}_P$,
    operator ${\mathbf{\Omega}}_P$ is sufficient for propagation
    because total fixpoints correspond to stable models
    \medskip
  \item<5-> \structure{Note}
    In addition to ${\mathbf{\Omega}}_P$, most ASP-solvers apply backward
    propagation, originating from program completion
    \\
    (although this is unnecessary from a formal point of view)
  \end{itemize}
\end{frame}
%----------------------------------------------------------------
%%% Local Variables:
%%% mode: latex
%%% TeX-PDF-mode: t
%%% TeX-master: "../asp"
%%% End:
