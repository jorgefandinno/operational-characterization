% ----------------------------------------------------------------------
\begin{frame}{Basic idea}
  \bigskip
  \begin{itemize}
  \item \structure{Idea} \ Extend consequence operator to normal logic programs
    \medskip
  \item \structure{Logical background} \
    The idea is to turn a program's completion\\ into an operator such that
    \smallskip
    \begin{itemize}\normalsize
    \item the head atom of a rule must be $\mathit{true}$\\
      if the rule's body is $\mathit{true}$
    \item an atom must be $\mathit{false}$\\
      if the body of each rule having it as head is $\mathit{false}$
    \end{itemize}
  \end{itemize}
  \nocite{fitting02a}
\end{frame}
% ----------------------------------------------------------------------
\begin{frame}{Definition}
  Let $P$ be a normal logic program
  \medskip
  \begin{itemize}
  \item<2-> Define
    \[
    {\mathbf{\Phi}}_P\langle T,F \rangle
    =
    \langle
    {\mathbf{T}}_P\langle T,F \rangle,
    {\mathbf{F}}_P\langle T,F \rangle
    \rangle
    \]
  \item<3-> []
    where
    \begin{align*}
      {\mathbf{T}}_P\langle T,F \rangle
      & =
      \{\, \head{r} \mid r \in P, \pbody{r} \subseteq T, \nbody{r} \subseteq F \,\}
      \\\pause[4]
      {\mathbf{F}}_P\langle T,F \rangle
      & =
      \{\, a \in \atom{P} \mid
      \pbody{r} \cap F \neq \emptyset
      \text{ or }
      \nbody{r} \cap T \neq \emptyset
      \\
      & \kern 71pt
      \text{ for each } r \in P
      \text{ such that }
      \head{r} = a
      \,\}
    \end{align*}
  \item<5-> \structure{Note} \ We refer to ${\mathbf{\Phi}}_P$ as the \alert{Fitting operator} of program $P$
  \end{itemize}
\end{frame}
% ----------------------------------------------------------------------
\begin{frame}{Example}
  \bigskip
  \begin{itemize}
  \item<1->
    \(
    P
    =
    \left\{
      \begin{array}{lll}
        a \leftarrow                \quad &
        c \leftarrow a, \neg d      \quad &
        e \leftarrow b, \neg f
        \\
        b \leftarrow \neg a         \quad &
        d \leftarrow \neg c, \neg e \quad &
        e \leftarrow e
      \end{array}
    \right\}
    \)
    \bigskip
  \item<2-> Let's iterate ${\mathbf{\Phi}}_{P}$ on $\langle \{a\}, \{d\}\rangle$
    \[
    \begin{array}{rcl}
    {\mathbf{\Phi}}_{P}\langle\{a\}  ,\{d\}    \rangle &=&\langle\{a, c\} ,\{b,f\}  \rangle
    \\
    {\mathbf{\Phi}}_{P}\langle\{a,c\},\{b,f\}  \rangle &=&\langle\{a\}    ,\{b,d,f\}\rangle
    \\
    {\mathbf{\Phi}}_{P}\langle\{a\}  ,\{b,d,f\}\rangle &=&\langle\{a,c\}  ,\{b,f\}  \rangle
    \\
    & \vdots &
    \end{array}
    \]
  \end{itemize}
\end{frame}
% ----------------------------------------------------------------------
\begin{frame}{Fitting semantics}
  \bigskip
  \begin{itemize}
  \item<1-> Define the iterative variant of ${\mathbf{\Phi}}_P$ for $i\geq 0$ as
    \[
      {\mathbf{\Phi}}_P^0\langle T, F \rangle = \langle T, F \rangle
      \qquad\qquad
      {\mathbf{\Phi}}_P^{i+1}\langle T, F \rangle =
      {\mathbf{\Phi}}_P{\mathbf{\Phi}}_P^i\langle T, F \rangle
    \]
  \item<2-> Define the \alert{Fitting semantics} of a normal logic program $P$\\
    as the partial interpretation
    \[
      \textstyle{\bigsqcup_{i \geq 0}} {\mathbf{\Phi}}_P^i \langle \emptyset, \emptyset \rangle
    \]
  \end{itemize}
\nocite{fitting85a}
\end{frame}
% ----------------------------------------------------------------------
\begin{frame}{Example}
  \bigskip
  \begin{itemize}
  \item<1->
    \(
    P
    =
    \left\{
      \begin{array}{lll}
        a \leftarrow                \quad &
        c \leftarrow a, \neg d      \quad &
        e \leftarrow b, \neg f
        \\
        b \leftarrow \neg a         \quad &
        d \leftarrow \neg c, \neg e \quad &
        e \leftarrow e
      \end{array}
    \right\}
    \)
  \bigskip
  \item<2-> Computing the Fitting semantics of $P$
    \[
    \begin{array}{rclcl}
      \mathbf{\Phi}^0\langle\emptyset,\emptyset\rangle&=&                                              & &\langle\emptyset,\emptyset\rangle
      \\
      \mathbf{\Phi}^1\langle\emptyset,\emptyset\rangle&=&\mathbf{\Phi}\langle\emptyset,\emptyset\rangle&=&\langle\{a\}    ,\{f\}    \rangle
      \\
      \mathbf{\Phi}^2\langle\emptyset,\emptyset\rangle&=&\mathbf{\Phi}\langle\{a\}    ,\{f\}    \rangle&=&\langle\{a\}    ,\{b,f\}  \rangle
      \\
      \mathbf{\Phi}^3\langle\emptyset,\emptyset\rangle&=&\mathbf{\Phi}\langle\{a\}    ,\{b,f\}  \rangle&=&\langle\{a\}    ,\{b,f\}  \rangle%
      \\[10pt]\pause[3]
      \bigsqcup_{i \geq 0}\mathbf{\Phi}^i\langle\emptyset,\emptyset\rangle&=&\langle \{a\}, \{b,f\} \rangle
    \end{array}
    \]
  \end{itemize}
\end{frame}
% ----------------------------------------------------------------------
\begin{frame}{Properties}
  \bigskip
  Let $P$ be a normal logic program
  \medskip
  \begin{itemize}
  \item ${\mathbf{\Phi}}_P\langle \emptyset, \emptyset \rangle$ is monotonic
    \smallskip

    That is,
    \(
    {\mathbf{\Phi}}_P^i\langle \emptyset, \emptyset \rangle
    \sqsubseteq
    {\mathbf{\Phi}}_P^{i+1}\langle \emptyset, \emptyset \rangle
    \)
    \smallskip
  \item The Fitting semantics of $P$ is
    \begin{itemize}\normalsize
    \item not conflicting
    \item generally not total
    \end{itemize}
  \end{itemize}
\end{frame}
% ----------------------------------------------------------------------
\begin{frame}{Fitting fixpoints}
  \bigskip
  Let $P$ be a normal logic program,\\ and
  let $\langle T,F \rangle$ be a partial interpretation
  \medskip
  \begin{itemize}
  \item<1->
    $\langle T,F \rangle$ is a \alert{Fitting fixpoint} of $P$,
    if
    ${\mathbf{\Phi}}_P\langle T,F \rangle = \langle T,F \rangle$
  \smallskip
  \item<2-> []
    \begin{itemize}\normalsize
    \item the Fitting semantics is the least Fitting fixpoint
      \smallskip
    \item other Fitting fixpoints extend the Fitting semantics
      \smallskip
    \item total Fitting fixpoints correspond to supported models
    \end{itemize}
  \end{itemize}
\end{frame}
% ----------------------------------------------------------------------
\begin{frame}{Example}
  \bigskip
  \begin{itemize}
  \item<1->
    \(
    P
    =
    \left\{
      \begin{array}{lll}
        a \leftarrow                \quad &
        c \leftarrow a, \neg d      \quad &
        e \leftarrow b, \neg f
        \\
        b \leftarrow \neg a         \quad &
        d \leftarrow \neg c, \neg e \quad &
        e \leftarrow e
      \end{array}
    \right\}
    \)
    \bigskip
  \item<2-> $P$ has three total Fitting fixpoints
    \smallskip
    \begin{itemize}\normalsize
    \item<3-> $\langle \{a,c\},  \{b,d,e,f\} \rangle$
    \item<3-> $\langle \{a,d\},  \{b,c,e,f\} \rangle$
    \item<3-> $\langle \{a,c,e\},\{b,d,f\}   \rangle$
    \end{itemize}
    \medskip
  \item <4-> $P$ has three supported models, two of them are stable models
  \end{itemize}
\end{frame}
% ----------------------------------------------------------------------
\begin{frame}{Properties}

  Let ${\mathbf{\Phi}}_P\langle T,F \rangle=\langle T',F'\rangle$
  for\\
  a normal logic program $P$ and
  a partial interpretation $\langle T,F \rangle$
  \smallskip
  \begin{itemize}
  \item<2-> ${\mathbf{\Phi}}_P$ is \alert{stable model preserving}
  \item<2-> [] If $X$ is a stable model of $P$ such that $T\subseteq X$ and $X\cap F=\emptyset$,
    \\ then $T'\subseteq X$ and $X\cap F'=\emptyset$
    \smallskip
  \item<3-> \structure{Note} \ ${\mathbf{\Phi}}_P$ can be used for approximating stable models
    \\ and so for propagation in ASP solvers
    \medskip
  \item <4->  \structure{Disclaimer} \ ${\mathbf{\Phi}}_P$ is still insufficient because total fixpoints
    correspond to supported models not necessarily stable models
    \begin{itemize}\normalsize
    \item the problem is the same as with program completion
    \item the missing piece is non-circularity of derivations
    \end{itemize}
  \end{itemize}
\end{frame}
% ----------------------------------------------------------------------
\begin{frame}{Example}
  \bigskip
  \begin{itemize}
  \item<1->
    \(
    P
    =
    \left\{
      \begin{array}{lcl}
        a &\leftarrow & b \\
        b &\leftarrow & a
      \end{array}
    \right\}
    \)
    \bigskip
  \item<2-> Computing the Fitting semantics of $P$
    \\
    \medskip
    \(
    \begin{array}{lcl}
      \Phi_P^0\langle \emptyset,\emptyset\rangle &=&\langle \emptyset,\emptyset\rangle\\
      \Phi_P^1\langle \emptyset,\emptyset\rangle &=&\langle \emptyset,\emptyset\rangle
      \\[5pt]
      \bigsqcup_{i \geq 0}\mathbf{\Phi}^i\langle\emptyset,\emptyset\rangle&=&\langle\emptyset,\emptyset\rangle
    \end{array}
    \)
    \bigskip
  \item<3-> \structure{Observation} \
    Fitting semantics cannot assign $\mathit{false}$ to $a$ and $b$
    \\
    although they can never become $\mathit{true}$
  \end{itemize}
\end{frame}
% ----------------------------------------------------------------------
%
%%% Local Variables:
%%% mode: latex
%%% TeX-master: "../../main"
%%% End:
