% ----------------------------------------------------------------------
\begin{frame}{Well-founded operator}
  Let $P$ be a normal logic program\\
  and let $\langle T,F \rangle$ be a partial interpretation
  \medskip
  \begin{itemize}
  \item<2-> \structure{Observation} \
    Condition 1 (in the definition of an unfounded set)
    corresponds to  ${\mathbf{F}}_P\langle T,F \rangle$
    of ${\mathbf{\Phi}}_P\langle T,F \rangle$
    \medskip
  \item<3-> \structure{Idea} \
    Extend Fitting's operator $\mathbf{\Phi}_P$
  \item<4-> []
    \begin{itemize}\normalsize
    \item keep    ${\mathbf{T}}_P\langle T,F \rangle$ from ${\mathbf{\Phi}}_P\langle T,F \rangle$ and
    \item replace ${\mathbf{F}}_P\langle T,F \rangle$ from ${\mathbf{\Phi}}_P\langle T,F \rangle$ by
      ${\mathbf{U}}_P\langle T,F \rangle$
    \end{itemize}
    \medskip
  \item<only@5 > \structure{Note} \ An atom is $\mathit{false}$ if it belongs to the greatest unfounded set
  \item<only@7-> \structure{Definition} \
    \(
    {\mathbf{\Omega}}_P\langle T,F \rangle
    =
    \langle {\mathbf{T}}_P\langle T,F \rangle, {\mathbf{U}}_P\langle T,F \rangle\rangle
    \)
  \item<only@8->[] We refer to ${\mathbf{\Omega}}_P$ as the \alert{well-founded operator} of program $P$
    \smallskip
  \item<only@9-> \structure{Property} \
    \(
    {\mathbf{\Phi}}_P  \langle T,F \rangle
    \sqsubseteq
    {\mathbf{\Omega}}_P\langle T,F \rangle
    \)
  \end{itemize}
\end{frame}
% ----------------------------------------------------------------------
\begin{frame}{Example}
  \bigskip
  \begin{itemize}
  \item<1->
    \(
    P
    =
    \left\{
      \begin{array}{lll}
        a \leftarrow                \quad &
        c \leftarrow a, \neg d      \quad &
        e \leftarrow b, \neg f
        \\
        b \leftarrow \neg a         \quad &
        d \leftarrow \neg c, \neg e \quad &
        e \leftarrow e
      \end{array}
    \right\}
    \)
    \bigskip
  \item<2-> Let's iterate ${\mathbf{\Omega}}_P$ on $\langle \{c\}, \emptyset\rangle$
    \[
    \begin{array}{rcl}
      {\mathbf{\Omega}}_{P}\langle\{c\}  ,\emptyset  \rangle &=&\langle\{a\}  ,\{d,f\}     \rangle
      \\
      {\mathbf{\Omega}}_{P}\langle\{a\}  ,\{d,f\}    \rangle &=&\langle\{a,c\},\{b,e,f\}   \rangle
      \\
      {\mathbf{\Omega}}_{P}\langle\{a,c\},\{b,e,f\}  \rangle &=&\langle\{a\}  ,\{b,d,e,f\} \rangle
      \\
      {\mathbf{\Omega}}_{P}\langle\{a\}  ,\{b,d,e,f\}\rangle &=&\langle\{a,c\},\{b,e,f\}   \rangle
      \\
      & \vdots &
    \end{array}
    \]
  \end{itemize}
\end{frame}
% ----------------------------------------------------------------------
\begin{frame}{Well-founded semantics}
  \bigskip
  \begin{itemize}
  \item<1-> Define the iterative variant of ${\mathbf{\Omega}}_P$ for $i\geq 0$ as
    \[
    {\mathbf{\Omega}}_P^0\langle T, F \rangle = \langle T, F \rangle
    \qquad\qquad
    {\mathbf{\Omega}}_P^{i+1}\langle T, F \rangle =
    {\mathbf{\Omega}}_P{\mathbf{\Omega}}_P^i\langle T, F \rangle
    \]
  \item<2-> Define the \alert{well-founded semantics} of a normal logic program $P$\\
    as the partial interpretation
    \[
    \textstyle{\bigsqcup_{i \geq 0}} {\mathbf{\Omega}}_P^i \langle \emptyset, \emptyset \rangle
    \]
  \end{itemize}
  \nocite{gerosc91a}
\end{frame}
% ----------------------------------------------------------------------
\begin{frame}{Example}
  \bigskip
  \begin{itemize}
  \item<1->
    \(
    P
    =
    \left\{
      \begin{array}{lll}
        a \leftarrow                \quad &
        c \leftarrow a, \neg d      \quad &
        e \leftarrow b, \neg f
        \\
        b \leftarrow \neg a         \quad &
        d \leftarrow \neg c, \neg e \quad &
        e \leftarrow e
      \end{array}
    \right\}
    \)
  \bigskip
  \item<2-> Computing the well-founded semantics of $P$
    \[
    \begin{array}{rclcl}
      \mathbf{\Omega}^0\langle\emptyset,\emptyset\rangle&=&                                                & &\langle\emptyset,\emptyset\rangle
      \\
      \mathbf{\Omega}^1\langle\emptyset,\emptyset\rangle&=&\mathbf{\Omega}\langle\emptyset,\emptyset\rangle&=&\langle\{a\}    ,\{f\}    \rangle
      \\
      \mathbf{\Omega}^2\langle\emptyset,\emptyset\rangle&=&\mathbf{\Omega}\langle\{a\}    ,\{f\}    \rangle&=&\langle\{a\}    ,\{b,e,f\}\rangle
      \\
      \mathbf{\Omega}^3\langle\emptyset,\emptyset\rangle&=&\mathbf{\Omega}\langle\{a\}    ,\{b,e,f\}\rangle&=&\langle\{a\}    ,\{b,e,f\}\rangle
      \\[10pt]\pause[3]
      \bigsqcup_{i \geq 0}\mathbf{\Omega}^i\langle\emptyset,\emptyset\rangle&=&\langle\{a\},\{b,e,f\}\rangle
    \end{array}
    \]
  \end{itemize}
\end{frame}
% ----------------------------------------------------------------------
\begin{frame}{Properties}
  \bigskip
  Let $P$ be a normal logic program
  \medskip
  \begin{itemize}
  \item ${\mathbf{\Omega}}_P^i\langle \emptyset, \emptyset \rangle$ is monotonic
    \smallskip

    That is,
    \(
    {\mathbf{\Omega}}_P^i\langle \emptyset, \emptyset \rangle
    \sqsubseteq
    {\mathbf{\Omega}}_P^{i+1}\langle \emptyset, \emptyset \rangle
    \)
    \smallskip
  \item The well-founded semantics of $P$ is
    \begin{itemize}\normalsize
    \item not conflicting
    \item generally not total
    \end{itemize}
    \smallskip
  \item We have \
    \(
    \bigsqcup_{i \geq 0}
    {\mathbf{\Phi}}_P^i \langle \emptyset, \emptyset \rangle
    \sqsubseteq
    \bigsqcup_{i \geq 0} {\mathbf{\Omega}}_P^i \langle
    \emptyset, \emptyset \rangle
    \)
  \end{itemize}
\end{frame}
% ----------------------------------------------------------------------
\begin{frame}{Well-founded fixpoints}
  \bigskip
  Let $P$ be a normal logic program,\\ and
  let $\langle T,F \rangle$ be a partial interpretation
  \medskip
  \begin{itemize}
  \item<1->
    $\langle T,F \rangle$ is a \alert{well-founded fixpoint} of $P$,
    if
    ${\mathbf{\Omega}}_P\langle T,F \rangle = \langle T,F \rangle$
  \medskip
  \item<2-> []
    \begin{itemize}\normalsize
    \item the well-founded semantics is the least well-founded fixpoint
      \smallskip
    \item other well-founded fixpoints extend the well-founded semantics
      \smallskip
    \item \alert<3>{total well-founded fixpoints correspond to stable models}
    \end{itemize}
  \end{itemize}
\end{frame}
% ----------------------------------------------------------------------
\begin{frame}{Example}
  \bigskip
  \begin{itemize}
  \item<1->
    \(
    P
    =
    \left\{
      \begin{array}{lll}
        a \leftarrow                \quad &
        c \leftarrow a, \neg d      \quad &
        e \leftarrow b, \neg f
        \\
        b \leftarrow \neg a         \quad &
        d \leftarrow \neg c, \neg e \quad &
        e \leftarrow e
      \end{array}
    \right\}
    \)
    \bigskip
  \item<2-> $P$ has two total well-founded fixpoints
    \smallskip
    \begin{itemize}\normalsize
    \item<3-> $\langle \{a,c\}, \{b,d,e,f\} \rangle$
    \item<3-> $\langle \{a,d\}, \{b,c,e,f\} \rangle$
    \end{itemize}
    \medskip
  \item <4-> Both of them represent stable models of $P$
  \end{itemize}
\end{frame}
% ----------------------------------------------------------------------
\begin{frame}{Properties}
  Let ${\mathbf{\Omega}}_P\langle T,F \rangle=\langle T',F'\rangle$
  for\\
  a normal logic program $P$ and
  a partial interpretation $\langle T,F \rangle$
  \smallskip
  \begin{itemize}
  \item<2-> ${\mathbf{\Omega}}_P$ is \alert{stable model preserving}
  \item<2-> [] If $X$ is a stable model of $P$ such that $T\subseteq X$ and $X\cap F=\emptyset$,
    \\ then $T'\subseteq X$ and $X\cap F'=\emptyset$
    \smallskip
  \item<3-> \structure{Note} \ ${\mathbf{\Omega}}_P$ can be used for approximating stable models
    \\ and so for propagation in ASP solvers
    \smallskip
  \item <4-> [] % In contrast to ${\mathbf{\Phi}}_P$,
    Operator ${\mathbf{\Omega}}_P$ is sufficient for propagation
    because total fixpoints correspond to stable models
    \medskip
  \item<5-> \structure{Note} \
    Most ASP solvers apply --- in addition to ${\mathbf{\Omega}}_P$ --- \alert{backward propagation},
    originating from program completion
    \\
    (although unnecessary from a formal point of view)
  \end{itemize}
\end{frame}
% ----------------------------------------------------------------------
\begin{frame}{Well-founded operator reloaded}{Alternative definition}
  Let $P$ be a normal logic program\\
  and let $(L,U)$ with $L\subseteq U$ be a partial interpretation
  \bigskip
  \begin{itemize}
  \item <2-> Define
    \[
      {\mathbf{\Omega}}_P(L,U)
      =
      (\Cn{\reduct{P}{U}},\Cn{\reduct{P}{L}})
    \]
  \item <3-> \structure{Properties}
    \begin{itemize}\normalsize
    \item The least fixpoint  of ${\mathbf{\Omega}}_P$ corresponds to the well-founded semantics of $P$
      \smallskip
    \item Total fixpoints of ${\mathbf{\Omega}}_P$ correspond  to the stable models of $P$
    \end{itemize}
  \end{itemize}
  \nocite{truszczynski18a}
\end{frame}
% ----------------------------------------------------------------------
%
%%% Local Variables:
%%% mode: latex
%%% TeX-master: "../../main"
%%% End:
